%!TEX root=bare_conf.tex
\section{Experimental Setup}\label{sec:setup}
To evaluate the projection approach proposed in Section~\ref{sec:approach}, 
this paper compares our tool---PML\_Checker with three open-source tools which represent the existing different static detection approaches. 
%In the rest of this section, this paper presents detailed information of the test subjects (Section~\ref{ssec:ts}), selected approaches to compare with (Section~\ref{ssec:ca}), the parameters and metrics (Section~\ref{ssec:pm}) in our experiments.
\subsection{Test Tools}\label{ssec:tool}
In our experiment, there are three state-of-the-art detection tools compared with PML\_Checker: CppCheck\footnote{CppCheck. trac.cppcheck.net/wiki. 2016.}, Splint\footnote{Splint. http://www.splint.org/. 2010.} and RL\_Detector. They are all open-source static detection tools, and have a good performance in detecting memory leaks.

Splint develops different detecting standards for different types of errors in the compilation process, it detects memory leaks by annotations to record the pointer object’s lifetime~\cite{EL02}. CppCheck first creates symbol database of variables, functions and so on, and then it detects memory leaks by the embedded inspection classes. RL\_Detector is a static detection tool implemented, which adopts resources streamlined slices construction and is achieved a comparative accurate analysis of memory leaks. This section compares our tool PML\_Checker which implements the presented approach with the above tools.
\subsection{Test Subjects}\label{ssec:ts}
\begin{table}[!h]
\newcommand{\tabincell}[2]{\begin{tabular}{@{}#1@{}}#2\end{tabular}}
\center
\caption{Information of real world repositories}\label{tab:1}
\begin{tabular}{|c|p{2.3cm}<{\centering}|p{2.3cm}<{\centering}|}
\hline
\textbf{Statistical metrics} & \textbf{SPEC CPU $2000$} & \textbf{ SIR  }\\
\hline
 \tabincell{c}{The number of\\test program} & 15 & 15\\
\hline
 \tabincell{c}{Total number of\\lines(kloc)} & 581 &267.9\\
\hline
 \tabincell{c}{The average number\\of lines(kloc)} & 38.7 & 17.9\\
\hline
 \tabincell{c}{The line number of\\biggest program(kloc)} &	205.8 &	122.2\\
\hline
\end{tabular}
\end{table}
Table~\ref{tab:1} lists the information of real world repositories. This experiment adopts two real world repositories for evaluation: SPEC CPU 2000\footnote{SPEC. http://www.spec.org/cpu/. 2007.} and SIR\footnote{SIR. http://sir.unl.edu/content/sir.php. 2017.} (Software-artifact Infrastructure Repository). Specifically, we take $15$ programs with $581$ thousand lines code from SPEC CPU $2000$, and  $15$ programs with $267.9$ thousand lines code from SIR. However, we can not count the false negatives of the two repositories, because of the large amount of code. 
%
\begin{table}[!h]
\newcommand{\tabincell}[2]{\begin{tabular}{@{}#1@{}}#2\end{tabular}}
\center
\caption{Information of artificial cases}\label{tab:2}
\begin{tabular}{|c|p{0.8cm}<{\centering}|p{0.8cm}<{\centering}|p{0.8cm}<{\centering}|p{0.8cm}<{\centering}|}
\hline
\multirow{2}{*}{\tabincell{c}{\textbf{Statistical}\\\textbf{metrics}}}&\multicolumn{2}{|c|}{\textbf{CC code}} &                              \multicolumn{2}{|c|}{\textbf{SARD cases}}\\\cline{2-5}
& \tabincell{c}{\textbf{control}\\\textbf{flow}} & \tabincell{c}{\textbf{data}\\\textbf{type}}    &  \tabincell{c}{\textbf{bad}} &  \tabincell{c}{\textbf{good}}\\
\hline
 \tabincell{c}{The number of\\test program} & 10 & 10 & 20 & 20\\
\hline
\tabincell{c}{Total number of\\lines(Loc)} &148  & 326 & 793&	821\\
\hline
 \tabincell{c}{The average number\\of lines(Loc)} &14.8 & 32.6 & 39.7 & 41.2\\
\hline
 \tabincell{c}{The line number of\\biggest program(Loc)}  &18 & 95 &75 &77\\
\hline
\end{tabular}
\end{table}
%

In order to count false negatives of PML\_Checker, this experiment also uses artificial small code as test cases: SARD\footnote{NIST. https://samate.nist.gov/SARD/. 2016.} (Software Assurance Reference Dataset) and CC code\protect\footnote{PML\_Checker.https://github.com/sunxiaohuiczcz/PML\_Checker.git.2017.}. SARD is a public benchmark providing test cases. This experiment take $40$ memory-related test cases, including $20$ bad cases with memory errors and $20$ corresponding good cases with no errors. CC code is provided by this paper to verify PML\_Checker from two aspects: complex control flows and complex data types.
A complex control flow refers to a control flow graph with more than one control flow branches, which is the focus of this paper. To make the experiment more convincing, this paper considers the complexity of data types. In our experiment, complex data types include data structure like linked list, struct, array and the combination of these data types. CC code presents $10$ small programs with different complex control flows, and $10$ small programs with different complex data types. Adhering to the principle of single case coverage scenarios minimized\footnote{Test Case Design. http://ecomputernotes.com/software-engineering/test-case-design. 2017.}, each case only covers one test scenario and includes only one memory leak.
The detailed information of study cases are shown in Table~\ref{tab:2}.
\subsection{Metrics}\label{ssec:m}
There are seven metrics in our experiments for evaluation.
\begin{itemize}
\item \textit{TW:} The total number of the reported memory leaks.
\item \textit{TL:} The total number of memory leaks contained in a program, and it takes the number of pointers pointing to memory blocks as the measurement standard.
\item \textit{NF:} The number of false positives.
\item \textit{MLF:} The difference of \textit{TW} and \textit{NF}, which denotes the number of real memory leaks in the test results.
\item \textit{FP:} The false positive rate, a metric used to measure the accuracy. It can be calculated by the following formula: \textit{FP}=$\frac{\textit{NF}}{\textit{TW}}$.
\item \textit{NP:} The false negative rate, a metric used to measure the accuracy. It can be calculated by the following formula: \textit{NP}=1-$\frac{\textit{TW}}{\textit{TL}}$.
\item \textit{RunTime:} The run time of each tool for C program.
\end{itemize}